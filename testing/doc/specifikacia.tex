\documentclass[12pt,a4paper]{article}
\usepackage[utf8]{inputenc}
\usepackage{slovak}
\begin{document}
\title{WebcamWizard}
\author{Ladislav Bačo, Anton Kovaľ \\ {\it Kubo Kováč (vedúci)}}
\maketitle

\paragraph{Cieľ projektu:} Vybrané možnosti ovládania PC pomocou webkamery.

\paragraph{Princíp:} Snímanie a rozoznávanie užívateľských gest pomocou webkamery, pričom na uľahčenie identifikácie budeme používať farebné LED diódy. Táto technológia môže mať široké využitie v hernom priemysle, školstve, ale môže priniesť aj nové koncepty ovládania elektrických spotrebičov.\\
V ročníkovom projekte by sme sa chceli zaoberať interaktívnou tabuľou, pričom pracovná plocha tabule bude zobrazovaná pomocou dataprojektora a užívateľ bude snímaný webkamerou z notebooku alebo externou webkamerou.\\
Naša interaktívna tabuľa by mala mať nasledujúce vlastnosti:
\begin{enumerate}
\item kreslenie pomocou \uv{svetelného pera}
\item kalibrácia pracovnej plochy vzhľadom na aktuálne podmienky
\item možnosť zmeny farby pera
\item uloženie aktuálnej pracovnej plochy ako bitmapový obrázok
\item načítanie bitmapového obrázka
\end{enumerate}

(V prípade príliš veľkých problémov s interaktívnou tabuľou sa budeme zaoberať hrou so skutočným priestorovým efektom.)

\paragraph{Implementácia:} programovací jazyk C++, knižnica OpenCV (prípadne ešte OpenGL)
\paragraph{Podporované platformy:} Linux, Windows XP, Vista, 7, možno aj Mac OS X
\paragraph{repozitár:} {\tt git://github.com/laciKE/WebcamWizard.git}


\end{document}
